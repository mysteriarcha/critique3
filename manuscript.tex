% !TeX program = lualatex


\documentclass[a4paper, 9pt]{article}
\usepackage[utf8]{inputenc}
\usepackage{fontspec}
\usepackage[english]{babel}

%\setmainfont{Times New Roman}
% Load blindtext package for dummy text
\usepackage{blindtext}
% Load the setspace package
\usepackage{setspace}
% Using \doublespacing in the preamble 
% changes the text to double-line spacing
\doublespacing

\usepackage[style=apa]{biblatex} %Imports biblatex package
\addbibresource{refs.bib} %Import the bibliography file

\usepackage{graphicx} % Required for inser\usepackage{multicol}ting images
\usepackage{multirow}
\usepackage{hyperref}
\hypersetup{
	colorlinks=true,
	linkcolor=blue,
	filecolor=magenta,      
	urlcolor=cyan,
	citecolor=cyan,
	pdfpagemode=FullScreen,
}
\usepackage{multicol}
\usepackage{microtype}
\usepackage{amsmath}
\usepackage{easyReview} % for \comment, \alert, etc.
\usepackage{fancyhdr}
\usepackage[paper=a4paper, inner=2cm, outer=2cm, top=2cm, bottom=1.5cm]{geometry}
\usepackage{paralist}

%\usepackage[switch, modulo]{lineno}
%\linenumbers	
%\modulolinenumbers[1]

% \setlength{\columnsep}{1cm}
\pagestyle{fancy}

\renewcommand{\footnoterule}{\noindent\smash{\rule[3pt]{\textwidth}{.4pt}}}
\newcommand{\refs}{(\alert{REFS})}

\usepackage{authblk}

\title{Critique of Critical Criticism: a review of uses and abuses of theory criticisms in ecology and economics}
\author[1]{Mysteriarcha}
\author[1]{Carlos P. Small}
\affil[1]{Vienna Circle Club}

\begin{document}
		
	\maketitle
	
	\begin{abstract}
		Ecology has, like other branches of life sciences, a bipolar 
		pattern of research traditions, from mechanicism-reductionism
		to emergentism-holism. Each tradition has pitfalls and 
		problems of its own. However these are compounded when they 
		are involved in sociopolitical issues, and borrowings from 
		other sciences (particularly from economics) are seen as a 
		legitimization mechanism for ill-founded theses. Here we 
		review classic and current debates on ecological paradigms 
		and their sociopolitical implications and borrowings from 
		economic theory, and discuss their soundness. Furthermore we 
		point to alternatives in formal economic tools, and as a 
		case study provide a historical analysis of the economic 
		interests during the early and largely unknown development 
		of succession theory in forestry. We do this by calling for 
		much more historical and philosophical consciousness for 
		researchers when ``problematic" interdisciplinary 
		relationships are detected.
	\end{abstract}
	
	
	
\end{document}